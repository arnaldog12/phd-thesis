\section*{\centering{ABSTRACT}}

The face is considered the primary biometric trait for machine-readable travel documents such as passports. In this context, the \icao standard defines a set of photographic requirements to ensure the image quality and simplify the face recognition process. However, because the number of requirements defined by such a standard is high (almost 30), the compliance verification of a single face image is still a challenge. Usually, problems with multiple tasks, such as the \icao requirements, are broken into independent subproblems that are solved separately and then recombined. Nevertheless, it ignores the common information between related tasks and increases the risk of overfitting. \acf{mtl} has proven to be an important technique for solving multiple related tasks simultaneously. It explores the common and distinct aspects of tasks from the same domain to improve the generalization among all tasks. In addition, MTL focuses on learning a useful representation that can yield benefits, particularly in scenarios where a labeled dataset for a task is limited. Finally, in the case of \aclp{dnn}, MTL can help reduce the number of parameters and inference speed. This research proposes the first deep \acl{mtl} method designed for automatic evaluation of both photographic and geometric requirements of the \icao standard, called \methodname. Undercomplete Autoencoders are extended to employ a multi-and-collaborative learning approach, in which both supervised and unsupervised learning are performed concurrently and in a collaborative manner. The method is trained using an ad hoc image dataset and evaluated using an official benchmark system that is also used by other approaches presented in the literature. The experiments show the method proposed achieves the best results in terms of Equal Error Rate for 9 out of the 23 photographic requirements of \icao, which was not achieved by any other individual method evaluated. Therefore, the proposed method can be considered the best overall solution among academic works published in the literature and private SDKs. Overall, the median Equal Error Rate (3.3\%) is also competitive. In terms of running time, the proposed method stands out among the fastest methods to evaluate all 23 requirements according to the official benchmark. On the other hand, there are space for improvements on results of eye's landmark location and some specific requirements that may require additional investigation. Finally, through Neural Network visualization techniques, relevant patterns related to the requirements of the \icao standard could be observed. 

\vspace{2em}

\noindent
\textbf{Keywords}: Face Quality, ICAO, \icao, \acl{mtl}, Autoencoders, \acl{dl}

\newpage
