\section*{\centering{ABSTRACT}}

The face is considered the primary biometric trait for machine-readable travel documents, like passports. In this context, the \icao standard defines a set of photographic requirements to ensure image quality and simplify the face recognition process. However, the assessment of face image compliance to the ISO/ICAO standard is still predominantly performed by humans today due to the lack of automatic evaluation systems to perform this task. Therefore, this research aims to propose a method to assess all photographic requirements of the \icao standard that can achieve competitive results with the state-of-the-art in terms of \acf{eer}. It employs a multi-and-collaborative learning approach, where supervised and unsupervised learning are performed concurrently and collaboratively. The method is trained using an ad hoc image dataset and evaluated by an official \textit{benchmark} system also used by other approaches presented in the literature. Preliminary experimental evidence presented in this thesis proposal shows that our method achieved satisfactory results. In comparison to the specialized literature, the proposed method was able to obtain the best \acs{eer} results in 9 out of the 23 requirements. Overall, the median \acs{eer} (3.3\%) is also competitive. However, the mean \acs{eer} (9.3\%) was significantly influenced by the worst results. In terms of CPU running time, the proposed method stands out among the fastest. In addition, through Neural Network visualization techniques, we could observe relevant patterns related to the requirements of the \icao standard. For the conclusion of the doctoral research, investigations will be carried out regarding (i) improvements to the dataset, (ii) faster and more robust pre-processing methods for mixed image sizes, (iii) different architectures and cost functions, and (iv) inclusion of eye location prediction.

\vspace{2em}

\noindent
\textbf{Keywords}: ICAO, \icao, \acl{mtl}, Autoencoders, \acl{dl}, \acl{cnn}

\newpage
