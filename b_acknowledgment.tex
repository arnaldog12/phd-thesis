% \section*{\centering{ACKNOWLEDGMENT}} 
\selectlanguage{portuguese}
\section*{\centering{AGRADECIMENTOS}}

\textbf{A meus orientadores.} A \textit{Herman Martins} por ter me aceitado no doutorado após vários anos tentando e sendo recusado por outros orientadores. Também por contribuir significantemente nas nossas reuniões sempre com suas ideias, questionamentos e incentivo. A \textit{Leonardo Vidal} por me acompanhar desde o 2º período da graduação, passando pelo PET, 2 PIBICs, Mestrado e ter-me feito apaixonar pela área da Computação que mais amo até hoje: Visão Computacional. Considero-o como o meu pai acadêmico.

\textbf{A minha família.} A meu pai, \textit{Arnaldo Gualberto da Silva}, por nunca medir esforços para me dar todas as oportunidades que tive na vida. Também por me ensinar sobre responsabilidade e me mostrar que nada resiste ao trabalho. À minha mãe, \textit{Luzia de Andrade}, por construir os pontos mais importantes da minha personalidade e caráter. E por nunca ter me deixado desistir de nada na vida. A meu irmão, \textit{Juan Gualberto}, pelo tudo que vivemos como irmãos e pelo orgulho que tem de mim - sentimento recíproco. Ao contrário da frase popular, costumo dizer que somos mais do que irmãos, somos amigos.

\textbf{A Sabrina Figueiredo.} Por me apoiar em todas as decisões e me permitir levá-las adiante. Se arrisquei, errei, ou acertei, é porque eu sabia que em todas as ocasiões você estaria lá. Sem você, eu não teria feito outro doutorado e essa tese não teria sido escrita. Você é meu porto seguro, GPS, e combustível. Também é minha força, motivação e inspiração. Eu me encontro em você pra ser o melhor que sou. 

\textbf{A Theo Gualberto.} Por mudar a maneira como eu via o mundo, minhas prioridades e me fazer uma pessoa melhor em alguns aspectos. Também por ser minha Rede Neural Biológica. Espero que em algum momento você entenda, filho, que tudo que fiz (mesmo antes de você nascer) e ainda farei de importante na vida é para que, um dia, você tenha orgulho do seu pai.

\newpage
\selectlanguage{english}