% \section*{\centering{ACKNOWLEDGMENT}} 
\section*{\centering{AGRADECIMENTOS}}

Agradeço à mim. Obrigado. De nada. 

\newpage
