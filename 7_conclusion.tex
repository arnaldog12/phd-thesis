\section{Concluding Remarks}

This work presented a deep learning-based method developed to evaluate photographic requirements of \icao standard, called \methodname. We extend undercomplete Convolutional Autoencoders to employ a multi-learn approach in a collaborative fashion. The architecture has three main components: (i) an encoder to encode the input image in a useful 256-D representation shared by (ii) an unsupervised branch to reconstruct the input image; and (iii) a supervised branch to assess the requirements as a multi-label problem. 

We evaluated our method using a small amount of unbalanced data. It is composed by a subset of the \ficvtest dataset in conjunction with an ad-hoc dataset built especially for ICAO requirements. The network was trained from scratch and a custom loss function was used in the network optimization process. It balances the two different tasks solved by the method, i.e., regression (image reconstruction) and multi-label classification (requirements assessment). Additionally, the training was monitored to preserve the model with highest F-beta score.

Individually, the \methodname was able to achieve considerable results. In training, most of the metrics evaluated has a score greater than 90\%. Through the analysis of these metrics, we were able to notice interesting pattern about the method predictions. For example, it is better predicting the positive class, but the \aclp{fp} are more troublesome than \aclp{fn}. Additionally, in the FICV competition, the method presented substantial performance in most of the requirements. Nevertheless, three requirements had unacceptable performance in terms of \acs{eer} ($> 40\%$) and will require particular treatment to improve their results.

In comparison to other methods evaluated by the FICV benchmark, the \methodname was able to achieve state-of-art results in 9 out of 23 requirements with a median EER of 3.3\%. Therefore, the proposed work has the highest amount of best results in a single method compared to all the works presented in the literature or as private SDK tools. In terms of EER, the \methodname has the second-best median EER compared to methods that evaluate all requirements. Furthermore, the proposed method is also the fastest one to evaluate all requirements on CPU, taking only 1.8s to evaluate an input image. However, our running time still has a spot for improvement since it is highly influenced by the face detector used as preprocessing. The architecture proposed by itself takes only 0.15s to run in the CPU.

As future works, we intend to concentrate efforts in three different aspects to improve the results:

\begin{itemize}
\item \textbf{Dataset}: the dataset quality can be improved by increasing (i) the number of images and (ii) the variability of patterns of some requirements (like hat/cap). Thereby, the network can learn more effective descriptors and decrease the EER in these requirements. The most unbalanced requirements may require a special attention and, probably, more images must be gathered. Furthermore, the dataset labels may be revised to fix possible labeling errors.

\item \textbf{Preprocessing}: as discussed in Chapter \ref{sec:results}, the preprocessing step may have been responsible for some of the errors. First, we can replace the current face detector to a more fast and reliable approach like \cite{faceboxes}. It can help decreasing the detection time (approximately 90\% of total running time) and the Rejection Rates (0.4\% max). Moreover, we must improve the input image provided to the network. For some images, the cropping and resizing steps are either removing or generating artifacts that can harm the network learning. See Figures \ref{fig:variedbgd} and \ref{fig:hairacrosseyes} of section \ref{sec:ficv_results} for further details. Thus, we need to find a better way to preprocess the input image as whole without injuring the trade-off between speed and accurate results.

\item \textbf{Method}: some elements of the network can also be considered. It includes, but is not limited to, the architecture and the loss function. For example, the Capsule Neural Networks (CapsNets), proposed by \cite{sabour2017dynamic}, can be used to leverage the hierarchical relationship between the requirements. Recent techniques like Self-Supervised Learning \citep{doersch2017multi} may also be considered. Finally, we intend to test another loss functions, specially designed for the multi-label classification task (e.g., the Contrastive Loss \citep{khosla2020supervised}).
\end{itemize}

We also have plans to add another branch to our network to predict the \citeReq{\eyecenterlocation} requirement evaluated by FICV competition. It may require the adjustment of the loss function in use and the need of labeled samples.

\begin{table}[hb]
\centering
\caption{Schedule of activities.}
\label{tab:schedule}
\begin{tabular}{c|c|l|c|l|l|l|c|}
\cline{2-8}
 & \multicolumn{2}{c|}{\textbf{2021}} & \multicolumn{4}{c|}{\textbf{2022}} & \textbf{2023} \\ \cline{2-8} 
 & Q3 & \multicolumn{1}{c|}{Q4} & Q1 & \multicolumn{1}{c|}{Q2} & \multicolumn{1}{c|}{Q3} & \multicolumn{1}{c|}{Q4} & Q1 \\ \hline
\multicolumn{1}{|c|}{\textbf{Activity 1}} & \multicolumn{1}{l|}{} & \cellcolor[HTML]{333333} & \multicolumn{1}{l|}{\cellcolor[HTML]{333333}} & \cellcolor[HTML]{333333} & \cellcolor[HTML]{333333} & \cellcolor[HTML]{333333} & \multicolumn{1}{l|}{\cellcolor[HTML]{333333}} \\ \hline
\multicolumn{1}{|c|}{\textbf{Activity 2}} & \cellcolor[HTML]{333333} & \multicolumn{1}{c|}{\cellcolor[HTML]{333333}} &  & \multicolumn{1}{c|}{} & \multicolumn{1}{c|}{} & \multicolumn{1}{c|}{} &  \\ \hline
\multicolumn{1}{|c|}{\textbf{Activity 3}} &  & \multicolumn{1}{c|}{\cellcolor[HTML]{333333}} &  & \multicolumn{1}{c|}{} & \multicolumn{1}{c|}{} & \multicolumn{1}{c|}{} &  \\ \hline
\multicolumn{1}{|c|}{\textbf{Activity 4}} & \multicolumn{1}{l|}{} &  & \multicolumn{1}{l|}{\cellcolor[HTML]{333333}{\color[HTML]{333333} }} &  &  &  & \multicolumn{1}{l|}{} \\ \hline
\multicolumn{1}{|c|}{\textbf{Activity 5}} & \multicolumn{1}{l|}{} &  & \multicolumn{1}{l|}{\cellcolor[HTML]{333333}} & \cellcolor[HTML]{333333} & \cellcolor[HTML]{333333} & \cellcolor[HTML]{333333} & \multicolumn{1}{l|}{} \\ \hline
\multicolumn{1}{|c|}{\textbf{Activity 6}} & \multicolumn{1}{l|}{\cellcolor[HTML]{333333}} & \cellcolor[HTML]{333333} & \multicolumn{1}{l|}{\cellcolor[HTML]{333333}} & \cellcolor[HTML]{333333} & \cellcolor[HTML]{333333} & \cellcolor[HTML]{333333} & \multicolumn{1}{l|}{\cellcolor[HTML]{333333}} \\ \hline
\multicolumn{1}{|c|}{\textbf{Activity 7}} &  & \multicolumn{1}{c|}{} &  & \multicolumn{1}{c|}{} & \multicolumn{1}{c|}{} & \multicolumn{1}{c|}{\cellcolor[HTML]{333333}} & \cellcolor[HTML]{333333} \\ \hline
\end{tabular}
\end{table}

Lastly, the table \autoref{tab:schedule} present the schedule of main activities to be developed until the end of the this Doctoral research. The activities are described as follows:

\begin{itemize}
\item \textbf{Activity 1}: Bibliography Review.
\item \textbf{Activity 2}: Review of dataset labels.
\item \textbf{Activity 3}: Analysis of different face detectors.
\item \textbf{Activity 4}: Research and development of new pre-processing method.
\item \textbf{Activity 5}: 
\item \textbf{Activity 6}: Writing of technical-scientific reports and articles for academic publication
\item \textbf{Activity 6}: Write and defense of Doctoral research
\end{itemize}
