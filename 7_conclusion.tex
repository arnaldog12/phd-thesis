\section{Concluding Remarks}

This work presented a deep learning-based method developed to evaluate photographic requirements of \icao standard, called \methodname. We extend undercomplete Convolutional Autoencoders with a supervised branch that performs multi-label classification in a collaborative fashion with unsupervised learning. The network was trained from scratch using a small amount of data. 

The \methodname was able to achieve state-of-art results in 9 out of 23 requirements with a median EER of 3,3\%. Therefore, the proposed work has the highest amount of best results in a single method compared to all methods presented in the literature or as private SDK tools. In terms of EER, the \methodname has the second-best median EER compared to methods that evaluate all requirements. Furthermore, the proposed method is also the fastest one to evaluate all requirements on CPU, taking only 1.8s to evaluate an input image. However, our running time still has a spot for improvement since it is highly influenced by the face detector used as preprocessing. The architecture proposed by itself takes only 0.15s to run in the CPU.

As future work, we intend to change the face detector of our preprocessing step to a faster and more reliable approach like \cite{faceboxes}. The goal is to decrease rejection rates and increase the number of images per second the proposed method can evaluate in CPU and GPU. Additionally, the dataset can be improved by adding more images and increasing the variability of some requirements, especially those that yielded the worst results. We also want to add another branch to our network to predict the \citeReq{\eyecenterlocation} requirement evaluated by the Biolab-ICAO framework in the FICV competition. Finally, we intend to test another loss functions, specially designed for the multi-label classification task (e.g., the Contrastive Loss \citep{khosla2020supervised}).

