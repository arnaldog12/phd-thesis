\section*{\centering{RESUMO}}

A face é considerada o principal traço biométrico para documentos de viagem legíveis por máquina, como passaportes. Neste contexto, o padrão \icao define um conjunto de requisitos fotográficos para garantir a qualidade da imagem e simplificar o processo de reconhecimento facial. No entanto, a avaliação da conformidade da imagem facial com o padrão ISO/ICAO ainda é realizada principalmente por humanos devido à falta de sistemas de avaliação automática para realizar esta tarefa. Neste trabalho, apresentamos um método baseado em aprendizado multitarefa profundo, denominado \methodname, projetado para avaliação automática dos requisitos fotográficos do padrão \icao. Nossa rede emprega uma abordagem de multi-aprendizagem colaborativa, onde a aprendizagem supervisionada e não-supervisionada são realizadas simultaneamente e de forma cooperativa. Evidências experimentais apresentadas nesta tese mostram que nosso método atingiu os melhores resultados em 9 dos 23 requisitos relacionados à face da norma \icao, o que não foi alcançado por nenhum outro método avaliado individualmente. Portanto, o método proposto pode ser considerado a melhor solução geral entre trabalhos acadêmicos publicados na literatura ou como SDKs privados. Em termos de Equal Error Rate (EER), a \methodname alcançou a segunda melhor mediana (3,3\%). Por fim, a implementação do método usando TensorFlow em C++ leva apenas 1,8 segundos para ser executada em CPU, sendo o método mais rápido a avaliar todos os 23 requisitos de acordo com um benchmark oficial.

\newpage
