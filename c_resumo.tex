\section*{\centering{RESUMO}}

\noindent
A face é considerada o principal traço biométrico para documentos de viagem legíveis por máquina, como passaportes. Neste contexto, o padrão \icao define um conjunto de requisitos fotográficos para garantir a qualidade da imagem e simplificar o processo de reconhecimento facial. No entanto, a avaliação da conformidade da imagem facial com o padrão ISO/ICAO ainda é realizada principalmente por humanos devido à falta de sistemas de avaliação automática que sejam completos e precisos o suficiente para execução desta tarefa. Diante deste cenário, a presente pesquisa propõe um método capaz de aferir todos os requisitos fotográficos do padrão \icao e alcançar resultados competitivos com o estado-da-arte em termos da Taxa de Erro Igual (do inglês, \textit{\acl{eer}}, \acs{eer}). Tal método emprega uma abordagem de multi-aprendizagem colaborativa, onde a aprendizagem supervisionada e não-supervisionada são realizadas simultaneamente e de forma cooperativa. O método é treinado usando um banco de imagens especialmente construído para o problema descrito e avaliado por um sistema de \textit{benchmark} oficial também utilizado por outras abordagens presentes na literatura. Evidências experimentais preliminares apresentadas nesta proposta de tese mostram que o método atingiu resultados satisfatórios. Quando comparado aos resultados da literatura especializada, o método proposto foi capaz de obter os melhores resultados de \acs{eer} em 9 dos 23 requisitos. De forma global, a \acs{eer} mediana (3,3\%) também mostra-se competitiva. Entretanto, a \acs{eer} média (9,3\%) foi significantemente influenciada pelos resultados dos piores requisitos. Em termos de tempo de execução em CPU, o método proposto destaca-se entre os mais rápidos. Além disso, através de técnicas de visualização de Redes Neurais, foi possível observar padrões relevantes relacionados aos requisitos da norma \icao. Para conclusão dessa pesquisa, serão realizadas investigações a respeito de (i) melhorias para o \textit{dataset}, (ii) métodos de pré-processamento mais rápidos e robustos a diversos tamanhos de imagem, (iii) diferentes arquiteturas e funções de custo, e (iv) inclusão da predição da localização dos olhos.

\vspace{2em}

\noindent
\textbf{Palavras-chave}: ICAO, \icao, \acl{mtl}, Autoencoders, \acl{dl}, \acl{cnn}

\newpage
