\selectlanguage{portuguese}
\section*{\centering{RESUMO}}

\noindent
A face é considerada o principal traço biométrico para documentos de viagem legíveis por máquina, como passaportes. Neste contexto, o padrão \icao define um conjunto de requisitos fotográficos para garantir a qualidade da imagem e simplificar o processo de reconhecimento facial. No entanto, devido ao grande número de requisitos definidos por esse padrão (quase 30), a verificação de conformidade de uma única imagem facial ainda é um desafio. Normalmente, problemas com várias tarefas, como os requisitos desse padrão, são divididos em subproblemas independentes que são resolvidos separadamente e, em seguida, recombinados. No entanto, isso ignora as informações comuns entre tarefas relacionadas e aumenta o risco de sobreajuste. O Aprendizado Multitarefa (do inglês, \acl{mtl}, \acs{mtl}) tem se provado uma técnica importante para resolver várias tarefas relacionadas simultaneamente. Ele explora os aspectos comuns e distintos de tarefas do mesmo domínio para melhorar a generalização entre todas as tarefas. Além disso, o \acs{mtl} concentra-se em aprender uma representação útil que possa gerar benefícios, especialmente em cenários em que um conjunto de dados rotulados para uma tarefa é limitado. Por fim, no caso das Redes Neurais Profundas, o \acs{mtl} pode ajudar a reduzir o número de parâmetros e a velocidade de inferência. Esta pesquisa propõe o primeiro método de aprendizado profundo multitarefa projetado para avaliação automática dos requisitos do padrão \icao, denominado \methodname. Autoencoders subcompletos são estendidos para empregar uma abordagem de multi-aprendizagem colaborativa, onde a aprendizagem supervisionada e não-supervisionada são realizadas simultaneamente e de forma cooperativa. O método é treinado usando um banco de imagens especialmente construído para o problema descrito e avaliado por um sistema de \textit{benchmark} oficial também utilizado por outras abordagens presentes na literatura. Os experimentos mostram que o método proposto alcança os melhores resultados em termos de Taxa de Erro Igual (do inglês, \acl{eer}, \acs{eer}) para 9 dos 23 requisitos fotográficos, o que não foi alcançado por nenhum outro método conforme a bibliografia consultada. Portanto, o método proposto pode ser considerado a melhor solução geral entre trabalhos acadêmicos publicados na literatura e SDKs privados analisados. No geral, a \acs{eer} mediana (3,3\%) também é competitiva. Em termos de tempo de execução, o método proposto se destaca entre os métodos mais rápidos para avaliar todos os 23 requisitos segundo o benchmark oficial. Por outro lado, há espaço para melhorias nos resultados da localização dos olhos e alguns requisitos específicos, que podem exigir investigação adicional. Por fim, por meio de técnicas de visualização de Redes Neurais, foram identificados padrões de representação relevantes aos requisitos do padrão \icao.

\vspace{1em}

\noindent
\textbf{Palavras-chave}: Qualidade Facial, ICAO, \icao, Aprendizado Multitarefa, Autoencoders, Aprendizado Profundo.

\newpage
\selectlanguage{english}
