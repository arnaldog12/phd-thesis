\section{Introduction}

The face has been traditionally used in identity documents for visual recognition of a person and therefore represents one of the most used physical traits for biometric recognition \citep{ferrara2012face}. It also has an essential role in many other biometric-related applications like video surveillance \citep{de2015partially} or facial expression recognition \citep{anil2016literature}. Compared to other physical traits, the face has some advantages. For instance, the acquisition of facial features via digital photography is non-intrusive, can be performed remotely, and does not require full cooperation from the individual or even specialized hardware.

% However, the use of face for recognition traditionally achieved lower accuracy when compared to other biometric traits like fingerprint and iris. Such results are influenced by the higher variability in capture conditions. According to \cite{hernandez2019faceqnet}, high levels of accuracy are obtained in face recognition in constrained scenarios, with collaborative users, and favorable acquisition conditions. Therefore, the capability of dealing with variability factors needed to be addressed.

In the context of biometric recognition via identity documents, many efforts have been made to allow machine-assisted verification of an individual's identity over the years. One of the most important projects was developed by \acf{icao}. In 2002, the \acs{icao} defined directives for automatic biometric recognition of people using machines \citep{icao2003report}. The goal was to detail the ideal conditions of facial images to perform robust and highly accurate face verification/recognition by machines. These regulations are followed by many countries around the globe, including, for instance, all member states of the European Union \citep{ebinger2008international}.

The \icao \citep{iso-iec} is an official standard that defines the requirements for facial photography used in electronic passports based on the \acs{icao} guidelines. It defines a set of quality rules that include: photographic properties (positioning, camera focus, etc.), scene constraints (lighting, pose, expression), and digital attributes (image resolution, image size, etc.). For example, a given facial image suitable for passports must have a frontal pose with a neutral expression, open eyes, no objects covering the face (e.g., hair or veil), uniform background, illumination, and focus.

Since the number of requirements defined by the ISO/ICAO standards is high (almost 30), the compliance verification of a single face image is still a challenge to be accomplished. According to \cite{ferrara2012multi}, this task is still performed visually by human experts, sometimes with the support of an automated system. It prevents agility in critical scenarios such as international airports, where this task is performed millions of times a day. Therefore, complete automation of this task is still an ongoing request and may help avoid the need for a human expert and accelerate the document production process. In this context, the following question arises: ``would it be possible to conceive a single Machine Algorithm that could efficiently (in terms of memory and running time) and precisely (with low error rates) assess all the requirements of \icao standard''?

In recent years, \acfp{dnn} have gained a prominent position in Computer Science due to their high propensity to recognize intricate patterns. One of the main advantages of this technique is the network's inherent ability to extract information from raw data, sometimes with little or no preprocessing. It allows a generic learning process that does not depend on attributes explicitly chosen or extracted from the data. In Computer Vision, this task is usually performed by \acfp{cnn}, a particular type of architecture initially developed for images as input, reducing the number of parameters and improving the training process \citep{goodfellow2016deep}.

\acf{mtl} is a specific \acl{ml} technique that allows multiple tasks to be solved at the same time by exploring familiar and different aspects between them \citep{zhang2017survey}. It goes against the traditional methodology where one task is learned at a time. Usually, significant problems - like the ICAO requirements - are broken into small, independent, and reasonable subproblems that are solved separately and then recombined. As stated in \cite{Caruana1997}, it is sometimes counterproductive since a potentially rich source of information available in many real-world problems is ignored: the information contained in other tasks from the same domain. In the context of Deep Learning, the \acs{mtl} can allow the network to share information of related tasks to improve generalization in all tasks. Moreover, the \acl{mtl} is best suited when there are limited training samples in multiple related tasks.

Compared to the case where each task of a multitask problem is solved individually by a specific network, the multitask networks present several advantages. First, the amount of parameters and memory used by the model is considerably reduced due to the inherent layer sharing. Secondly, the inference speed increases since such networks avoid the recomputation of features in the shared layers. Finally, such networks can improve performance if the related tasks share complementary information or act as a regularizer for one another \citep{vandenhende2021multi}. In the context of the \icao standard, the \acs{mtl} can solve all requirements in parallel, reducing the processing time and increasing the success rates.

In this thesis proposal, we present a Deep Learning-based method for automatic evaluation of the photographic requirements of the \icao standard, called \methodname. Our network is trained from scratch in a Multitask Learning approach and with limited data (approximately 5,700 images only). The architecture is based on Autoencoders, but the training is performed in unsupervised (image reconstruction) and supervised (multi-label classification) manners simultaneously. A preliminary experimental evaluation has shown that our method could outperform algorithms presented in the literature and SDK tools available as commercial solutions in the assessment of most requirements. We also apply \acf{shap} \citep{shap2018}, \acf{pca} \citep{pca}, and t-SNE \citep{tsne} to understand our network predictions related to each requirement presented in the \icao standard.

\subsection{Objectives}	

The general objective of this thesis proposal is to develop a state-of-art method for assessment of the requirements from \icao standard using \acl{mtl}. By state-of-art, we mean an algorithm that can assess all 23 requirements in a single method with the best median \acs{eer} among all published works. The specific goals are:

\begin{itemize}
\item Build an ad-hoc labeled database for evaluation of ICAO requirements;
\item Propose a method to perform preprocessing of face images such that the network can have accurate results without compromising run time speed;
\item Investigate and develop a method for localization of eyes centers;
\item Research and develop a single method for assessment of all photographic and pose-specific tests of \icao standard using \acf{mtl} and compare it against all methods published in the literature; and
\item Analyse the features learned by the proposed method and its outputs using dimensionality reduction and explanation techniques;
\end{itemize}

\subsection{Thesis Structure}

The rest of this thesis is structured as follows. First, we detail the fundamental concepts used in this work in Chapter \ref{sec:background}. In Chapter \ref{sec:literature}, a review of the principal \acl{mtl} architectures and works published for the \icao standard are presented. Next, in Chapter \ref{sec:method}, we describe the proposed methodology and its current stage of development. The preliminary evaluations over the proposed work obtained up to the moment are presented in Chapter \ref{sec:results}. Finally, we discuss the conclusions and propose a schedule for future works.
